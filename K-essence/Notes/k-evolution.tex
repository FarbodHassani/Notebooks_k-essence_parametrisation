

\documentclass[a4paper,10pt]{article}
\pdfoutput=1
\usepackage{jcappub}
\usepackage{bbold}

\usepackage[english]{babel}
\usepackage[utf8]{inputenc}
\usepackage{amsmath}
\usepackage{color}
\usepackage{amsfonts}
\usepackage{graphicx}
\usepackage{amssymb}
\usepackage{eufrak}
\usepackage{etoolbox}
\usepackage{amsmath}
\usepackage{empheq}
\usepackage{cancel}
\usepackage[most]{tcolorbox}
\usepackage{float}              % Activate [H] option to place figure HERE
\usepackage{listings}

\newtcbox{\mymath}[1][]{%
    nobeforeafter, math upper, tcbox raise base,
    enhanced, colframe=yellow!30!black,
    colback=yellow!30, boxrule=1pt,
    #1}
%%%%%%%%%%%%%%%%%%%%%%%%%%%%

\def\be{\begin{equation}}
\def\ee{\end{equation}}
\def\bea{\begin{eqnarray}}
\def\eea{\end{eqnarray}}
\def\bean{\begin{eqnarray*}}
\def\eean{\end{eqnarray*}}
\def\cd{\cdot}
\def\vp{\varphi}
\def\l {\langle}
\def\re {\rangle}
\def \dd {\partial}
\def \ra {\rightarrow}
\def \la {\lambda}
\def \La {\Lambda}
\def \De {\Delta}
\def \DH {\Delta_{\rm HI}}
\newcommand{\de}{\delta}
\def \b {\beta}
\def \al {\alpha}
\def \ka {\kappa}
\def \Ga {\Gamma}
\def \ga {\gamma}
\def \si {\sigma}
\def \Si {\Sigma}
\def \ep {\epsilon}
\def \om {\omega}
\def \Om {\Omega}
\def \lap {\triangle}
\def \ep {\epsilon}


%%%%%%%%%%%%%%%%%%%%%%%%%%%%%%%%%%%
%Special definitions for this paper
%%%%%%%%%%%%%%%%%%%%%%%%%%%%%%%%%%%

\newcommand{\MyRed}{\color [rgb]{0.8,0,0}}
\newcommand{\MyGreen}{\color [rgb]{0,0.7,0}}
\newcommand{\MyBlue}{\color [rgb]{0,0,0.8}}
\newcommand{\MyBrown}{\color [rgb]{0.8,0.4,0.1}}
\newcommand{\MyPurple}{\color [rgb]{0.6,0.0,0.6}}
\def\GV#1{{\MyRed [GV: #1]}}
\def\RD#1{{\MyGreen [RD:  {\tt #1}]}} 
\def\RDt#1{{\MyGreen #1}}   
\def\GM#1{{\MyBlue [GM: #1]}}  
\def\GF#1{{\MyPurple [GF: #1]}}    



\newcommand{\ie}{\emph{i. e.}}
\newcommand{\cf}{\emph{cf.}}
\newcommand{\etal}{\emph{et al.}\xspace}
\newcommand{\eg}{\emph{e. g.}}

\newcommand{\Scal}{\mathcal S}
\newcommand{\DD}{\mathcal D}
\newcommand{\EE}{\mathcal E}
\newcommand{\MM}{\mathcal M}
\newcommand{\HH}{\mathcal H}

\newcommand{\Real}{\mathbb{R}}
\newcommand{\bn}{\boldsymbol{n}}
\newcommand{\bv}{\boldsymbol{v}}
\newcommand{\bx}{\boldsymbol{x}}
\newcommand{\bnabla}{\boldsymbol{\nabla}}
\newcommand{\bell}{\boldsymbol{\ell}}
\newcommand{\bal}{\boldsymbol{\alpha}}


%Farbod commands
\newcommand{\Ge}{G_{\text{eff}}}
\newcommand{\MP}{M_{\text{pl}}}
\newcommand{\PP}{\mathcal{P}}

%%%%%%%%%%%%%%%%%%%%%%%%%%%%%%%%%%%%%%%%%%



\title{... }

\author[a]{.}
\author[b]{, .}
\author[c]{, .}
\author[d]{,.}
\author[e]{,.}

%\affiliation[a]{
%Universit\'e de Gen\`eve, D\'epartement de Physique Th\'eorique and CAP,
%24 quai Ernest-Ansermet, CH-1211 Gen\`eve 4, Switzerland
%}

%\emailAdd{farbod.hassani@unige.ch}
%\emailAdd{martin.kunz@unige.ch}
\emailAdd{..}
\emailAdd{..}

\abstract{
}

\begin{document}
\maketitle
%%%%%%%%%%%%%%%%%
%%%INTRODUCTION
%%%%%%%%%%%%%%%%%
\section{Relativistic Poisson equation}
The conformal Newtonian gauge (also known as the longitudinal gauge) reads \url{https://arxiv.org/abs/astro-ph/9506072}-
\be
ds^2 = a^2 (\tau) \Big[-(1+2 \Psi) d\tau^2 +  (1-2 \Phi)dx_i dx^i  \Big]
\ee
The (0,0) component of  first order perturbed Einstein equation gives,
\be
k^2 \Phi + 3 \mathcal{H} \Big(\Phi' +  \HH \Psi\Big) =-4 \pi G a^2 \bar{\rho} \, \delta
\ee
The last equation is called relativistic Poisson equation which for small scales ($k \gg \HH$) and using the fact that $\Phi'$ is very suppressed in general we can retrieve the Newtonian Poisson equation, which reads as,
\be
k^2 \Phi = -4 \pi G a^2 \bar{\rho} \, \delta
\ee
 \subsection{Newtonian Poisson equation  in the k-evolution}
 Here we are going to compare $\PP_{\delta}$ from Poisson equation to the output one to see how much Newtonian Poisson equation is valid. We also can compute $k^2 \Phi  + 4 \pi G a^2 \bar{\rho} \delta$ but we need to take care of the signs. To obtain $\Phi$, $\delta$ and etc. we use the following relations 
 \be
 \PP _{\Psi} = \Psi_k ^2 \, , \;  \PP _{\delta} = \delta_k ^2
 \ee
 and also $\rho_m=\rho_m(0) /a^3$ and $4 \pi G \rho_{cr} =(3/2) {H_0}^2 $, so we have  $4 \pi G  \rho_m a^2 = 4 \pi G  \frac{\rho_m (a)}{\rho_{cr} (0)} \rho_{cr} (0) a^2 = 4 \pi G  \frac{\Omega_m (0) a^{-3} \rho_{cr}}{\rho_{cr} (0)} \rho_{cr} (0) a^2= (3/2) {H_0}^2   {\Omega_m (0) a^{-3} }  a^2 = ( 3/2) \mathcal{H}_0^2 \Omega_m (0) /a $
\\ Assuming that only matter clusters, which is not completely true in the case of k-essence we have,
\be
4 \pi G a^2 \bar{\rho} \delta  = ( 3/2) \mathcal{H}_0^2 \frac{\Omega_m (0) }{a} \sqrt{\PP_{\delta}}
\ee
We are going to compute $|\frac{k^2 \sqrt{\PP_{\Phi}}  - ( 3/2) \mathcal{H}_0^2 \frac{\Omega_m (0) }{a} \sqrt{\PP_{\delta}}} {k^2 \sqrt{\PP_{\Phi}}}|$ to test the Newtonian Poisson equation,
        \begin{figure}[H]
 \includegraphics[scale=0.55]{./Plot_Newt_001} 
 \end{figure}
 \subsection{Relativistic Poisson equation}
Now we want to check if we can reduce the error of Newtonian Poisson equation by considering relativistic terms as following. 
\be
\Big| \; \frac{k^2 \Phi + 3 \mathcal{H} \Big(\cancelto{0}{\Phi'} +  \HH \Psi\Big)    - ( 3/2) \mathcal{H}_0^2 {\Omega_m (0) } \sqrt{\PP_{\delta}} (1+z)    }{k^2 \Phi}  \;\Big| \approx 0
\ee
We neglect $\Phi'$ because it is almost suppressed. We get the following plot,
We expect to get "0" in the Gevolution by computing the last expression. 
%%%%%%%%%%%%%%%%%%%%%%%%%%%%%%%%%%%%%%%%%%%%%
     \begin{figure}[H]
 \includegraphics[scale=0.55]{./Plot_rel_001} 
 \end{figure}

 


%%%%%%%%%%%%%%%%%%%%%%%%%%%%%
\section{Conclusions}

\setcounter{equation}{0}
%%%%%%%%%%%%%%%%%%%%%

 
\section*{Acknowledgements}

We acknowledge financial support from the Swiss National Science Foundation.


%%%%%%%%%%%%%%%%%%%%%%%%%%%%%%%
%%%%%%%%%%%%%%%%%%%%%%%%%%%%%%%
\appendix
%
\bibliographystyle{JHEP}
%\bibliography{biblio}
\bibliography{EFTDE}

\end{document}

\begin{thebibliography}{999}
\newcommand{\bb}{\bibitem}

 \end{thebibliography}


\end{document}


