

\documentclass[a4paper,10pt]{article}
\pdfoutput=1
\usepackage{jcappub}
\usepackage{bbold}

\usepackage[english]{babel}
\usepackage[utf8]{inputenc}
\usepackage{amsmath}
\usepackage{color}
\usepackage{amsfonts}
\usepackage{graphicx}
\usepackage{amssymb}
\usepackage{eufrak}
\usepackage{etoolbox}
\usepackage{amsmath}
\usepackage{empheq}
\usepackage{cancel}
\usepackage[most]{tcolorbox}
\usepackage{float}              % Activate [H] option to place figure HERE

\newtcbox{\mymath}[1][]{%
    nobeforeafter, math upper, tcbox raise base,
    enhanced, colframe=yellow!30!black,
    colback=yellow!30, boxrule=1pt,
    #1}
%%%%%%%%%%%%%%%%%%%%%%%%%%%%

\def\be{\begin{equation}}
\def\ee{\end{equation}}
\def\bea{\begin{eqnarray}}
\def\eea{\end{eqnarray}}
\def\bean{\begin{eqnarray*}}
\def\eean{\end{eqnarray*}}
\def\cd{\cdot}
\def\vp{\varphi}
\def\l {\langle}
\def\re {\rangle}
\def \dd {\partial}
\def \ra {\rightarrow}
\def \la {\lambda}
\def \La {\Lambda}
\def \De {\Delta}
\def \DH {\Delta_{\rm HI}}
\newcommand{\de}{\delta}
\def \b {\beta}
\def \al {\alpha}
\def \ka {\kappa}
\def \Ga {\Gamma}
\def \ga {\gamma}
\def \si {\sigma}
\def \Si {\Sigma}
\def \ep {\epsilon}
\def \om {\omega}
\def \Om {\Omega}
\def \lap {\triangle}
\def \ep {\epsilon}


%%%%%%%%%%%%%%%%%%%%%%%%%%%%%%%%%%%
%Special definitions for this paper
%%%%%%%%%%%%%%%%%%%%%%%%%%%%%%%%%%%

\newcommand{\MyRed}{\color [rgb]{0.8,0,0}}
\newcommand{\MyGreen}{\color [rgb]{0,0.7,0}}
\newcommand{\MyBlue}{\color [rgb]{0,0,0.8}}
\newcommand{\MyBrown}{\color [rgb]{0.8,0.4,0.1}}
\newcommand{\MyPurple}{\color [rgb]{0.6,0.0,0.6}}
\def\GV#1{{\MyRed [GV: #1]}}
\def\RD#1{{\MyGreen [RD:  {\tt #1}]}} 
\def\RDt#1{{\MyGreen #1}}   
\def\GM#1{{\MyBlue [GM: #1]}}  
\def\GF#1{{\MyPurple [GF: #1]}}    



\newcommand{\ie}{\emph{i. e.}}
\newcommand{\cf}{\emph{cf.}}
\newcommand{\etal}{\emph{et al.}\xspace}
\newcommand{\eg}{\emph{e. g.}}

\newcommand{\Scal}{\mathcal S}
\newcommand{\DD}{\mathcal D}
\newcommand{\EE}{\mathcal E}
\newcommand{\MM}{\mathcal M}
\newcommand{\HH}{\mathcal H}

\newcommand{\Real}{\mathbb{R}}
\newcommand{\bn}{\boldsymbol{n}}
\newcommand{\bv}{\boldsymbol{v}}
\newcommand{\bx}{\boldsymbol{x}}
\newcommand{\bnabla}{\boldsymbol{\nabla}}
\newcommand{\bell}{\boldsymbol{\ell}}
\newcommand{\bal}{\boldsymbol{\alpha}}


%Farbod commands
\newcommand{\Ge}{G_{\text{eff}}}
\newcommand{\MP}{M_{\text{pl}}}

%%%%%%%%%%%%%%%%%%%%%%%%%%%%%%%%%%%%%%%%%%



\title{... }

\author[a]{.}
\author[b]{, .}
\author[c]{, .}
\author[d]{,.}
\author[e]{,.}

%\affiliation[a]{
%Universit\'e de Gen\`eve, D\'epartement de Physique Th\'eorique and CAP,
%24 quai Ernest-Ansermet, CH-1211 Gen\`eve 4, Switzerland
%}

%\emailAdd{farbod.hassani@unige.ch}
%\emailAdd{martin.kunz@unige.ch}
\emailAdd{..}
\emailAdd{..}

\abstract{
In order to verify GR with the upcoming surveys we need to compare the cosmological quantities inferred by General Relativity (GR) assumptions  with the ones with non-GR  assumptions. In other words we need to compute the cosmological quantities (like $f \sigma_8$ ) consistently in each theory of gravity, including the specific background and perturbations. On the other hand, the effect of different dark energy models on linear and non-linear scales is still unknown and is not studied consistently. We are going to modify the relativistic N-body code "Gevolution" to implement the general effect of modified gravity models on scalar perturbations. To parametrize the possible deviations from GR, we use two parameters $\mu(k,z)$ and $\gamma(k,z)$ and we will discuss about the validity of GR by probing the parametric space of $\gamma$ and $\mu$...
}

\begin{document}
\maketitle
%%%%%%%%%%%%%%%%%
%%%INTRODUCTION
%%%%%%%%%%%%%%%%%
\section{Introduction}
\section{Idea}
We start from the FRW metric in Poisson gauge,
\be
ds^2 = a^2(\eta) \left[  - e^{2 \Psi} d\eta^2-2 B_i dx^i dt + ( e^{-2 \Phi} \delta_{ij} + h_{ij} ) dx^i dx^j \right] 
\ee
For the GR gravity the Einstein's equations including short wave correction reads,
\be
(1+4 \Phi) \nabla^2 \Phi  -3 \mathcal{H} \Phi' -3 \mathcal{H} ^2 \Psi +\frac 3 2 \delta^{ij} \Phi_{,i} \Phi_{,j} = -4 \pi G a^2 \Big( T_0^0 - \bar{T} ^0_0 \Big) \label{T00}
\ee
\begin{align}
\Big( \delta^i_k \delta^j_l - \frac 1 3 \delta^{ij} \delta _{kl}  \Big) \Big[  \frac 1 2 h''_{ij} + \mathcal{H} h'_{ij} - \frac 1 2 \nabla^2 h_{ij} & + B'_{(ij)} + 2\mathcal H B_{(i,j)} + (\Phi-\Psi)_{,ij} -2 (\Phi- \Psi) \Phi_{,ij} + 2 \Phi_{,i} \Phi_{,j} + 4 \Phi \Phi_{,ij}   \Big]   \nonumber \\ &
 = 8 \pi G a^2 \Big(\delta_{ik} T^i_l -\frac13 \delta_{kl}  T_i^i \Big)  
\end{align}
%We basically can have fully non-linear structures while the metric perturbations remain small. The order of quantities which are consistent with our perturbative expansion is shown in the following table from Adamek .et.al $\text{arxiv}.1604.06065$
%   \begin{figure}[H]
%   \centering
% \includegraphics[scale=0.6]{Table} 
% \end{figure}
 To parametrize the effect of modified gravity theories we modify the equation \ref{T00} in Fourier space by $G \to \Ge =  \mu(k,z) G$, which modifies the Poisson equation. Moreover,
 \be
 \frac{\Phi}{\Psi } = \gamma (k,z)
  \ee
 Like what is done in \url{https://arxiv.org/pdf/1106.4543.pdf} or \url{http://aliojjati.github.io/MGCAMB/mgcamb.html}.\\
We are going to do N-body simulation for some interesting functional form of $\Ge$ and $\gamma$ and statistically compare the observables in $\Lambda$CDM and modified gravity theory models.\\
Questions: \\
\begin{itemize}
 \item why we are going to implement it in an N-body code?\\
 Since non-linearities are important  and probably  change the results.
  \item why in Gevolution?\\
  Since Newtonian N-body codes do not let us modify gravity by definition, so we need a relativistic code to be consistent. 
 \item why $\Ge$ and $\gamma$ parametrization?\\
- we don't know yet about the paramterziation, but I'm (FH) going to discuss with Lucas Lombriser arXiv:1608.00522v1 he is in our group and specialist in this field.
\end{itemize}

\section{$G_{\text{eff}}$ implementation in gevolution}
For the moment we take $\gamma(k,z)=1$ and implement  $G_{}$
\setcounter{equation}{0}

%%%%%%%%%%%%%%%%%%%%%%%%%%%%%%%%%%%%%%%%%%%%%

 


%%%%%%%%%%%%%%%%%%%%%%%%%%%%%
\section{Conclusions}

\setcounter{equation}{0}
%%%%%%%%%%%%%%%%%%%%%

 
\section*{Acknowledgements}

We acknowledge financial support from the Swiss National Science Foundation.


%%%%%%%%%%%%%%%%%%%%%%%%%%%%%%%
%%%%%%%%%%%%%%%%%%%%%%%%%%%%%%%
\appendix
%
\bibliographystyle{JHEP}
%\bibliography{biblio}
\bibliography{EFTDE}

\end{document}

\begin{thebibliography}{999}
\newcommand{\bb}{\bibitem}

 \end{thebibliography}


\end{document}


